%% abtex2-modelo-trabalho-academico.tex, v-1.9.6 laurocesar
%% Copyright 2012-2016 by abnTeX2 group at http://www.abntex.net.br/
%%
%% This work may be distributed and/or modified under the
%% conditions of the LaTeX Project Public License, either version 1.3
%% of this license or (at your option) any later version.
%% The latest version of this license is in
%%   http://www.latex-project.org/lppl.txt
%% and version 1.3 or later is part of all distributions of LaTeX
%% version 2005/12/01 or later.
%%
%% This work has the LPPL maintenance status `maintained'.
%%
%% The Current Maintainer of this work is the abnTeX2 team, led
%% by Lauro César Araujo. Further information are available on
%% http://www.abntex.net.br/
%%
%% This work consists of the files abntex2-modelo-trabalho-academico.tex,
%% abntex2-modelo-include-comandos and abntex2-modelo-references.bib
%%

% ------------------------------------------------------------------------
% ------------------------------------------------------------------------
% abnTeX2: Modelo de Trabalho Academico (tese de doutorado, dissertacao de
% mestrado e trabalhos monograficos em geral) em conformidade com
% ABNT NBR 14724:2011: Informacao e documentacao - Trabalhos academicos -
% Apresentacao
% ------------------------------------------------------------------------
% ------------------------------------------------------------------------

\documentclass[
	% -- opções da classe memoir --
	12pt,				% tamanho da fonte
%	openright,			% capítulos começam em pág ímpar (insere página vazia caso preciso)
%	twoside,			% para impressão em recto e verso. Oposto a oneside
    oneside,			% Oposto a twoside
	a4paper,			% tamanho do papel.
	% -- opções da classe abntex2 --
	%chapter=TITLE,		% títulos de capítulos convertidos em letras maiúsculas
	%section=TITLE,		% títulos de seções convertidos em letras maiúsculas
	%subsection=TITLE,	% títulos de subseções convertidos em letras maiúsculas
	%subsubsection=TITLE,% títulos de subsubseções convertidos em letras maiúsculas
	% -- opções do pacote babel --
	english,			% idioma adicional para hifenização
	french,				% idioma adicional para hifenização
	spanish,			% idioma adicional para hifenização
	brazil				% o último idioma é o principal do documento
	]{abntex2}

% ---
% Pacotes 
% ---
\usepackage{lmodern}			% Usa a fonte Latin Modern			
\usepackage[T1]{fontenc}		% Selecao de codigos de fonte.
\usepackage[utf8]{inputenc}		% Codificacao do documento (conversão automática dos acentos)
\usepackage{lastpage}			% Usado pela Ficha catalográfica
\usepackage{indentfirst}		% Indenta o primeiro parágrafo de cada seção.
\usepackage{color}				% Controle das cores
\usepackage{graphicx}			% Inclusão de gráficos
\usepackage{microtype} 			% para melhorias de justificação
\usepackage{filecontents}       % Used so that the external files can be placed in this file

\usepackage[brazilian,hyperpageref]{backref}	 % Paginas com as citações na bibl
\usepackage[alf]{abntex2cite}	% Citações padrão ABNT

% ---
% CONFIGURAÇÕES DE PACOTES
% ---

% ---
% Configurações do pacote backref
% Usado sem a opção hyperpageref de backref
\renewcommand{\backrefpagesname}{Citado na(s) página(s):~}
% Texto padrão antes do número das páginas
\renewcommand{\backref}{}
% Define os textos da citação
\renewcommand*{\backrefalt}[4]{
	\ifcase #1 %
		Nenhuma citação no texto.%
	\or
		Citado na página #2.%
	\else
		Citado #1 vezes nas páginas #2.%
	\fi}%
% ---
\newcommand{\basePath}{/var/www/html/baladecoco/monografia/latex}

% ---
% Informações de dados para CAPA e FOLHA DE ROSTO
% ---
\titulo{Modelo de Trabalho \\ Final de Graduação com \abnTeX}
\autor{Renan Marcel Ferreira Dias}
\local{Brasil}
\data{Junho de 2017}
\orientador{André Bernardi}
\instituicao{%
  Universidade Federal de Itajubá -- UNIFEI
  \par
  Instituto de Engenharia de Sistemas e Tecnologia da Informação
  \par
  Engenharia da Computação}
\tipotrabalho{Trabalho Final de Graduação}
% O preambulo deve conter o tipo do trabalho, o objetivo,
% o nome da instituição e a área de concentração
\preambulo{Monografia referente ao Trabalho Final de Graduação do curso de Engenharia da Computação da Universidade Federal de Itajubá}
% ---



% ---
% Configurações de aparência do PDF final

% alterando o aspecto da cor azul
\definecolor{blue}{RGB}{41,5,195}

% informações do PDF
\makeatletter
\hypersetup{
     	%pagebackref=true,
		pdftitle={\@title},
		pdfauthor={\@author},
    	pdfsubject={\imprimirpreambulo},
	    pdfcreator={LaTeX with abnTeX2},
		pdfkeywords={abnt}{latex}{abntex}{abntex2}{trabalho acadêmico},
		colorlinks=true,       		% false: boxed links; true: colored links
    	linkcolor=blue,          	% color of internal links
    	citecolor=blue,        		% color of links to bibliography
    	filecolor=magenta,      		% color of file links
		urlcolor=blue,
		bookmarksdepth=4
}
\makeatother
% ---

% ---
% Espaçamentos entre linhas e parágrafos
% ---

% O tamanho do parágrafo é dado por:
\setlength{\parindent}{1.3cm}

% Controle do espaçamento entre um parágrafo e outro:
\setlength{\parskip}{0.2cm}  % tente também \onelineskip

% ---
% compila o indice
% ---
\makeindex
% ---

% ----
% Início do documento
% ----
\begin{document}

% Seleciona o idioma do documento (conforme pacotes do babel)
%\selectlanguage{english}
\selectlanguage{brazil}

% Retira espaço extra obsoleto entre as frases.
\frenchspacing

% ----------------------------------------------------------
% ELEMENTOS PRÉ-TEXTUAIS
% ----------------------------------------------------------
% CAPA ---------------------------------------------------------------------- %
\imprimircapa

% FOLHA DE ROSTO ------------------------------------------------------------ %
\imprimirfolhaderosto*

% DEDICATORIA --------------------------------------------------------------- %
\begin{dedicatoria}
  %% Dedicatória

\vspace*{\fill}
\centering
\noindent
Dedicatória TODO.
\vspace*{\fill}

  % Dedicatória

\vspace*{\fill}
\centering
\noindent
Dedicatória TODO.
\vspace*{\fill}

\end{dedicatoria}

% AGRADECIMENTOS ------------------------------------------------------------ %
\begin{agradecimentos}
  % Dedicatória

Lorem ipsum dolor sit amet, consectetur adipiscing elit. Mauris est ipsum, lobortis vehicula scelerisque et, consequat in urna. Mauris aliquet placerat ex. Ut vel erat nisi. Nam in nisl pellentesque, tincidunt enim ut, efficitur justo. Vestibulum ante ipsum primis in faucibus orci luctus et ultrices posuere cubilia Curae; Duis id nulla consectetur, suscipit nisi sit amet, ornare lectus. Fusce fermentum, urna ac commodo varius, lectus ipsum commodo nulla, sit amet lacinia ipsum neque eget sem. Nunc pulvinar eros lectus, sit amet ornare metus semper vitae.

Sed dapibus metus sed velit vehicula, ac aliquet lorem lobortis. Nam nec elit justo. Morbi sollicitudin est mollis, rutrum sapien sed, luctus nisl. Praesent dignissim a dui quis finibus. Vestibulum sodales venenatis metus. Nulla lacinia elit sit amet iaculis congue. Proin pellentesque in orci laoreet blandit.

\end{agradecimentos}

% EPIGAFE ------------------------------------------------------------------- %
\begin{epigrafe}
  % Epigrafe

\vspace*{\fill}
\begin{flushright}
    \textit{``Não vos amoldeis às estruturas deste mundo, \\
    mas transformai-vos pela renovação da mente, \\
    a fim de distinguir qual é a vontade de Deus: \\
    o que é bom, o que Lhe é agradável, o que é perfeito.\\
    (Bíblia Sagrada, Romanos 12, 2)}
\end{flushright}

\end{epigrafe}

% RESUMO -------------------------------------------------------------------- %
\setlength{\absparsep}{18pt} % ajusta o espaçamento dos parágrafos do resumo
\begin{resumo}
  % Resumo

Segundo a \citeonline[3.1-3.2]{NBR6028:2003}, o resumo deve ressaltar o
objetivo, o método, os resultados e as conclusões do documento. A ordem e a extensão
destes itens dependem do tipo de resumo (informativo ou indicativo) e do
tratamento que cada item recebe no documento original. O resumo deve ser
precedido da referência do documento, com exceção do resumo inserido no
próprio documento. (\ldots) As palavras-chave devem figurar logo abaixo do
resumo, antecedidas da expressão Palavras-chave:, separadas entre si por
ponto e finalizadas também por ponto.

\textbf{Palavras-chave}: latex. abntex. editoração de texto.

\end{resumo}

% RESUMO EM INGLES ---------------------------------------------------------- %
\begin{resumo}[Abstract]
  % Resumo em ingles

\begin{otherlanguage*}{english}
   This is the english abstract.

   \vspace{\onelineskip}

   \noindent
   \textbf{Keywords}: latex. abntex. text editoration.
\end{otherlanguage*}

\end{resumo}

% ---
% inserir lista de ilustrações
% ---
\pdfbookmark[0]{\listfigurename}{lof}
\listoffigures*
\clearpage
% ---

% ---
% inserir lista de tabelas
% ---
\pdfbookmark[0]{\listtablename}{lot}
\listoftables*
\clearpage
% ---

% ---
% inserir lista de abreviaturas e siglas
% ---
\begin{siglas}
  \item[ABNT] Associação Brasileira de Normas Técnicas
  \item[abnTeX] ABsurdas Normas para TeX
\end{siglas}
% ---

% ---
% inserir o sumario
% ---
\pdfbookmark[0]{\contentsname}{toc}
\tableofcontents*
\clearpage
% ---

% ----------------------------------------------------------
% ELEMENTOS TEXTUAIS
% ----------------------------------------------------------
\textual
\chapter{Introdução}

% =========================================================================== %
\section{O Comércio Convencional}

Quando falava-se em comércio à 20 anos atrás, pensavamos em uma estabelecimento comercial, que podiamos nos deslocar até ele, entrar e sermos atendidos por um vendedor. Um lugar onde podiamos ver o produto que queriamos comprar, tocar, analisar sua cor e sentir seu cheiro. Após escolher, pagariamos com dinheiro ou cheque e levariamos para a casa o produto.

% =========================================================================== %
\section{A Internet}

O comércio muda muito sua cara de tempos em tempos, mas nunca tivemos um salto tão grande na forma de comprar e vender mercadorias como nos últimos 20 anos. Desde 1995 quando a Amazon, gigante americana do comércio eletrônico, fez sua primeira venda online, em uma época que o Brasil ainda estava conhecendo a internet, muita coisa mudou. Novas tecnologias nos permitem comprar e vender sem sair do conforto de nossas casas e também à pagar, receber e movimentar dinheiro por dispositívos móveis que estão conectados a internet 24h por dia, sem ter que ir ao banco e enfrentar filas.

Todas estas mudanças fizeram com que as empresas que quisesem continuar no mercado e ter sucesso nos tempos atuais, mudassem sua maneira de fazer negócio. A importância do comércio físico, apesar de grande, está em decadência, e a presença na web é um requisito fundamental. Empresas que são recordistas em vendas em lojas físicas, como por exemplo a Casa Bahia e o Magazine Luiza, investiram e ainda estão investindo pequenas fortunas na criação e manutenção de seus websites e aplicativos móveis.

% =========================================================================== %
\section{E-commerce}

Comércio eletrônico, e-business ou e-commerce são termos usados para definir qualquer tipo de negociação que envolva trasmissão de dados pela internet\cite{WhatIsEcommerce}. Vários tipos de negócios se encacham nesta definição, de sites de venda de camisetas à sistemas online de bancos onde o cliente pode fazer transações e contratar serviços online.

\subsection{Vantagens}

Uma loja online permite a uma empresa vender produtos com um preço diferenciado, beneficiando os compradores. Isso acontece pois seus gastos são menores, principalmente com vendedores e espaços físicos.

No modo tradicional de comércio, uma empresa que queira ter grande visibilidade, tem que desembolsar grandes quantidades de capital para se intalar em lugares estratégicos, onde o fluxo de compradores é grande. Na internet, um site e-commerce de uma empresa pequena tem tantas chances quanto o de uma empresa com maior capacidade financeira. Tudo depende da qualidade e segurança do site e da força do marketing. Com muito pouco capital, faz-se campanhas de marketing online que atingem muito mais pessoas por dia que uma loja física no melhor ponto de comércio do mundo, seja ele onde for. Isso acontece pois na internet não há barreiras geográficas, pode-se comprar e vender para qualquer lugar do mundo.

\begin{itemize}
  \item Facilidade e conforto de fazer compras sem sair de onde está.
  \item Maior disponibilidade de lojas e produtos.
  \item Melhores condições para pesquisa e comparação de preços.
  \item Sem filas ou espera por vendedores ou atendentes livres.
  \item Acesso a produtos de várias regiões do país e do mundo.
  \item Fácil acesso a promoções e cupons de desconto.
  \item Lojas abertas o tempo todo.
  \item 
\end{itemize}

\subsection{Desvantagens}

Nem tudo é perfeito no mundo do e-commerce. Salvo excessões, estas são algumas desvantagens deste tipo de negócio.

\begin{itemize}
  \item Está sujeito a falhas e pode ser vulnerável a ataques.
  \item Em transações com cartões de crédito, há o risco de roubo ou fraudes.
  \item Impossibilita a inspeção física do bem a ser adquirido.
  \item Preço e tempo de entrega podem inviabilizar o negócio.
  \item Dificuldade e demora no retorno ou troca de mercadorias
  \item Nem tudo pode ser vendido online, como por exemplo comidas perecíveis.
  \item Produtos de alto custo como jóias não tem segurança suficiente para serem despachados como encomenda.
  \item Não há o toque pessoal, interação entre cliente e comprador, que pode fazer diferença na concretização de uma venda.
\end{itemize}

% =========================================================================== %
\section{Drupal}

TODO Oq é Drupal
TODO Por que Drupal
TODO Comunidade

% =========================================================================== %
\section{Ambiente de Desenvolvimento}

TODO Dominio e Host
TODO Vagrant - Oq é, Por que, alternativas?
TODO Gulp - Oq é, Por que, alternativas?

% =========================================================================== %
\section{Tecnologias de Web}

\subsection{Back-end}

TODO Oq é back-end
TODO PHP
TODO MySQL

\subsection{Front-end}

TODO Oq é front-end
TODO CSS & LESS
TODO Javascript & jQuery

% =========================================================================== %
\section{SEO, Performance e Segurança}

\subsection{Search Engine Optimization}

TODO Oque é, pra que serve?

\subsection{Performance}

TODO Como medir, importancia?

\subsection{Segurança}

TODO Quais os riscos, quais medidas?
http://webscience.ie/blog/2010/security-issues-in-e-commerce/

\chapter{Objetivo}

\section{Objetivo I}

TODO

\section{Objetivo II}

TODO

\chapter{Método}

\section{Método I}

TODO

\section{Método II}

TODO

\chapter{Resultado}

\section{Resultado I}

TODO

\section{Resultado II}

TODO

\chapter{Discussão}

\section{Discussão I}

TODO

\section{Discussão II}

TODO

\chapter{Conclusão}

\section{Conclusão I}

TODO

\section{Conclusão II}

TODO


% ----------------------------------------------------------
% ELEMENTOS PÓS-TEXTUAIS
% ----------------------------------------------------------
\postextual
% ----------------------------------------------------------

% ----------------------------------------------------------
% Referências bibliográficas
% ----------------------------------------------------------
\bibliography{abntex2-modelo-references}

% ----------------------------------------------------------
% Glossário
% ----------------------------------------------------------
%
% Consulte o manual da classe abntex2 para orientações sobre o glossário.
%
%\glossary

% ----------------------------------------------------------
% Apêndices
% ----------------------------------------------------------

% ---
% Inicia os apêndices
% ---
\begin{apendicesenv}

% Imprime uma página indicando o início dos apêndices
\partapendices
% ----------------------------------------------------------
\chapter{Quisque libero justo}
% ----------------------------------------------------------
\lipsum[50]

\end{apendicesenv}
% ---



\end{document}
