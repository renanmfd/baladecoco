\chapter{Introdução}

% =========================================================================== %
\section{Comércio Convencional}

Quando falava-se em comércio à 20 anos atrás, pensavamos em uma estabelecimento comercial, que podiamos nos deslocar até ele, entrar e sermos atendidos por um vendedor. Um lugar onde podiamos ver o produto que queriamos comprar, tocar, analisar sua cor e sentir seu cheiro. Após escolher, pagariamos com dinheiro ou cheque e levariamos para a casa o produto.

TODO

% =========================================================================== %
\section{Internet}

O comércio muda muito sua cara de tempos em tempos, mas nunca tivemos um salto tão grande na forma de comprar e vender mercadorias como nos últimos 20 anos. Desde 1995 quando a Amazon, gigante americana do comércio eletrônico, fez sua primeira venda online, em uma época que o Brasil ainda estava conhecendo a internet, muita coisa mudou. Novas tecnologias nos permitem comprar e vender sem sair do conforto de nossas casas e também à pagar, receber e movimentar dinheiro por dispositívos móveis que estão conectados a internet 24h por dia, sem ter que ir ao banco e enfrentar filas.

Todas estas mudanças fizeram com que as empresas que quisesem continuar no mercado e ter sucesso nos tempos atuais, mudassem sua maneira de fazer negócio. A importância do comércio físico, apesar de grande, está em decadência, e a presença na web é um requisito fundamental. Empresas que são recordistas em vendas em lojas físicas, como por exemplo a Casa Bahia e o Magazine Luiza, investiram e ainda estão investindo pequenas fortunas na criação e manutenção de seus websites e aplicativos móveis.

% =========================================================================== %
\section{E-commerce}

Comércio eletrônico, e-business ou e-commerce são termos usados para definir qualquer tipo de negociação que envolva trasmissão de dados pela internet\cite{WhatIsEcommerce}. Vários tipos de negócios se encacham nesta definição, de sites de venda de camisetas à sistemas online de bancos onde o cliente pode fazer transações e contratar serviços online.

\subsection{Vantagens}

Uma loja online permite a uma empresa vender produtos com um preço diferenciado, beneficiando os compradores. Isso acontece pois seus gastos são menores, principalmente com vendedores e espaços físicos.

No modo tradicional de comércio, uma empresa que queira ter grande visibilidade, tem que desembolsar grandes quantidades de capital para se intalar em lugares estratégicos, onde o fluxo de compradores é grande. Na internet, um site e-commerce de uma empresa pequena tem tantas chances quanto o de uma empresa com maior capacidade financeira. Tudo depende da qualidade e segurança do site e da força do marketing. Com muito pouco capital, faz-se campanhas de marketing online que atingem muito mais pessoas por dia que uma loja física no melhor ponto de comércio do mundo, seja ele onde for. Isso acontece pois na internet não há barreiras geográficas, pode-se comprar e vender para qualquer lugar do mundo.

O fator tempo é também uma vantagem do e-business

\subsection{Desvantagens}

TODO Segurança
TODO Entrega

% =========================================================================== %
\section{Drupal}

TODO Oq é Drupal
TODO Por que Drupal
TODO Comunidade

% =========================================================================== %
\section{Ambiente de Desenvolvimento}

TODO Dominio e Host
TODO Vagrant - Oq é, Por que, alternativas?
TODO Gulp - Oq é, Por que, alternativas?

% =========================================================================== %
\section{Tecnologias de Web}

\subsection{Back-end}

TODO Oq é back-end
TODO PHP
TODO MySQL

\subsection{Front-end}

TODO Oq é front-end
TODO CSS & LESS
TODO Javascript & jQuery

% =========================================================================== %
\section{SEO, Performance e Segurança}

\subsection{Search Engine Optimization}

TODO Oque é, pra que serve?

\subsection{Performance}

TODO Como medir, importancia?

\subsection{Performance}

TODO Quais os riscos, quais medidas?
